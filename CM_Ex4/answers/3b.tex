Prove that if $\AB^* \in \bigC$ then $\lang_\bigC \notin co\REcl$

Proof: \\
Notice that $\AB^* \in \bigC \subset \REcl$, then $\AB^* \notin \overline{\bigC} = \REcl \backslash \bigC$. \\
Now, because $\overline{\bigC} \subseteq \REcl$, $\overline{\bigC} \neq \emptyset$ ($\bigC \neq \REcl$) and $\overline{\bigC} \neq \REcl$ ($\bigC \neq \emptyset$),
we get from the previous section that $\overline{ACC} \leq_m \lang_{\overline{\bigC}}$. \\
Hence $ACC \leq_m \overline{\lang}_{\overline{\bigC}}$ and then from $ACC \in \REcl \backslash \Rcl$ we get $ACC \notin co\REcl \Rightarrow \overline{\lang}_{ \overline{\bigC}} \notin co\REcl$. \\
It holds that: \\
$\overline{\lang}_{\overline{\bigC}} = \overline{\{\langle M \rangle : \lang(M) \in \overline{\bigC}\}} \\
    = \{\langle M \rangle : \langle M \rangle \text{ is not valid encoding of a TM }\} \cup \{\langle M \rangle : \lang(M) \in \bigC\}  \\
    = \{\langle M \rangle : \langle M \rangle \text{ is not valid encoding of a TM }\} \cup \lang_\bigC$ \\

Notice that $\lang_{M\_invalid} = \{\langle M \rangle : \langle M \rangle \text{ is not valid encoding of a TM }\} \in \Rcl$ \\
since we can build a TM that gets $w \in \AB^*$ and checks if it is a valid encoding $w = \langle M \rangle$, if it is ($w \notin \lang_{M\_invalid}$) - it rejects,
otherwise ($w \in \lang_{M\_invalid}$) it accepts.

So we got that: \\
1. $\overline{\lang}_{\overline{\bigC}} = \lang_{M\_invalid} \cup \lang_\bigC$. \\
2. $\lang_{M\_invalid} \in \Rcl \Rightarrow \lang_{M\_invalid} \in co\REcl$. \\
3. $\overline{\lang}_{\overline{\bigC}} \notin co\REcl$. \\

Since $co\REcl$ is closed to union (just like $\REcl$ is closed to union, proof is after this section) we get that $\lang_\bigC \notin co\REcl$, \\
because otherwise we will get from (1) and (2) that $\overline{\lang}_{\overline{\bigC}} \in co\REcl$ in contradiction with (3). \\

In conclusion, we get that $\lang_\bigC \notin co\REcl$. As required

\begin{center}
    \noindent\rule{4cm}{0.4pt}
\end{center}
\underline{Proof that $co\REcl$ is closed under union}: \\

Let there be $\lang_1, \lang_2 \in co\REcl$, we will show that $\lang = \lang_1 \cup \lang_2 \in co\REcl$. \\
$\lang_1, \lang_2 \in co\REcl$ so there are TM $M_1, M_2$ such that
$\lang(M_1) = \overline{\lang_1}, \lang(M_2) = \overline{\lang_2}$. \\
We shall define TM $M$ such that $\lang(M) = \overline{\lang}$:

$M$ on input $x$: \\
1. Run $M_1(x)$ and $M_2(x)$ in parallel (step by step). \\
2. If both of them accepts - accept. \\
3. Reject.

So we get: \\
If $x \in \overline{\lang} = \overline{\lang_1 \cup \lang_2} = \overline{\lang_1} \cap \overline{\lang_2}$ then
$x \in \overline{\lang_1} \wedge x \in \overline{\lang_2}$, so $M_1(x)$ and $M_2(x)$ accept $\Rightarrow M$ accepts. \\
If $x \notin \overline{\lang} = \overline{\lang_1 \cup \lang_2} = \overline{\lang_1} \cap \overline{\lang_2}$  then
$x \notin \overline{\lang_1} \vee x \notin \overline{\lang_2}$, so $M_1(x)$ or $M_2(x)$ reject or stuck in a loop: \\
- If $M_1(x)$ or $M_2(x)$ reject, then $M$ rejects. \\
- If $M_1(x)$ and $M_2(x)$ is stuck in a loop, then $M$ is stuck in a loop. \\

Indeed we get $\lang(M) = \overline{\lang}$. \\
So we get $\lang = \lang_1 \cup \lang_2 \in co\REcl$, as required. \\

