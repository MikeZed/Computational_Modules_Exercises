If $\lang \leq_m \overline{\lang}$, then $\lang \in \Rcl$ \\
The claim is false. \\

Proof: \\
Let us observe the following language:
\[
    \lang = \{0w: w \in ACC\} \cup \{1w: w \in \overline{ACC}\}
\]
First we will show that $\lang \notin co\REcl$ by showing the reduction - $ACC \leq_m \lang$.

We can define the reduction function $f: \AB^* \rightarrow \AB^*$ such that $\forall w \in \AB^*: f(w) = 0w$

Notice that $f$ is computable and $\forall w \in \AB^*$: \\
$w \in ACC \\
    \Longrightarrow f(w) = 0w \in \{0w: w \in ACC\} \\
    \Longrightarrow f(w) = 0w \in \lang$

$w \notin ACC \\
    \Longrightarrow f(w) = 0w \notin \{0w: w \in ACC\} \wedge f(w) = 0w \notin \{1w: w \in \overline{ACC}\} \\
    \Longrightarrow f(w) = 0w \notin \lang$

Hence we get $ACC \leq_m \lang$. \\
We saw in class that $ACC \in \REcl \backslash \Rcl \Rightarrow ACC \notin co\REcl$.
Therefore we get that $\lang \notin co\REcl \Rightarrow \underline{\lang \notin \Rcl}$. \\

Now we will show that the reduction $\lang \leq_m \overline{\lang}$ exists. \\
We can define function $g: \AB^* \rightarrow \AB^*$ such that $\forall x \in \AB^*$:
\[
    g(x) =
    \begin{cases}
        1w, & x = 0w \wedge w \in \AB^* \\
        0w, & x = 1w \wedge w \in \AB^* \\
        1,  & x = \empw                 \\
    \end{cases}
\]
Note - we assume w.l.o.g. that $\empw \notin ACC$, since it depends on
how we encode $\langle M, w \rangle$ for TM $M$.
In case $\empw \notin ACC$, we just need to change $g(x)$ to return $0$ for $x = \empw$. \\

Notice that $g$ is computable and $\forall x \in \AB^*$: \\
$x \in \lang \\
    \Longrightarrow x \in \{0w: w \in ACC\} \vee x \in \{1w: w \in \overline{ACC}\}$

If $x \in \{0w: w \in ACC\}
    \Longrightarrow g(x) \in \{1w: w \in ACC\}
    \Longrightarrow g(x) \notin \lang
    \Longrightarrow g(x) \in \overline{\lang}$

If $x \in \{1w: w \in \overline{ACC}\}
    \Longrightarrow g(x) \in \{0w: w \in \overline{ACC}\}
    \Longrightarrow g(x) \notin \lang
    \Longrightarrow g(x) \in \overline{\lang}$ \\

$x \notin \lang \\
    \Longrightarrow x \notin \{0w: w \in ACC\} \wedge x \notin \{1w: w \in \overline{ACC}\} \\
    \Longrightarrow x = \empw \vee x \in \{1w: w \in ACC\} \vee x \in \{0w: w \in \overline{ACC}\}$

If $x = \empw
    \Longrightarrow g(x) = 1 \in \{1w: w \in \overline{ACC}\}
    \Longrightarrow g(x) \in \lang
    \Longrightarrow g(x) \notin \overline{\lang}$

If $x \in \{1w: w \in ACC\}
    \Longrightarrow g(x) \in \{0w: w \in ACC\}
    \Longrightarrow g(x) \in \lang
    \Longrightarrow g(x) \notin \overline{\lang}$

If $x \in \{0w: w \in \overline{ACC}\}
    \Longrightarrow g(x) \in \{1w: w \in \overline{ACC}\}
    \Longrightarrow g(x) \in \lang
    \Longrightarrow g(x) \notin \overline{\lang}$ \\

Hence we get $\underline{\lang \leq_m \overline{\lang}}$. \\

In conclusion, we saw that $\lang \leq_m \overline{\lang}$ and $\lang \notin \Rcl$. \\
Hence the claim is false.