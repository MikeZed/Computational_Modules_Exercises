$\lang = \{\langle M \rangle : M \text{ is a TM and } |\lang(M)| > 1\}$ \\

We will show that $\lang \in \REcl \backslash \Rcl$. \\

Notice that $\lang = \lang_\bigC = \{\langle M \rangle : M \text{ is a TM and } \lang(M) \in \bigC\}$ for $\bigC = \{\lang \in \REcl : |\lang| > 1\}$. \\
It holds for $\bigC$ that: \\
1. $\emptyset \notin \bigC$ since $|\emptyset| = 0$ \\
2. $\bigC \neq \emptyset$ since $\lang = \{0, 1\} \in \bigC$ \\
3. $\bigC \subset \REcl$ since $\emptyset \notin \bigC \wedge \emptyset \in \REcl$

Hence, from Rice Theorem and its extension, we get that $\lang \notin co\REcl$, also meaning that $\underline{\lang \notin \Rcl}$. \\

Now we will show that $\lang \in \REcl$. \\
We shall build a TM $M'$ that accepts $\lang$:

$M'$ on input $\langle M \rangle$: \\
1. Check whether $\langle M \rangle$ is a valid encoding of a TM. \\
2. If it is not - reject. \\
3. Let $w_1, w_2, ...$ be a lexicographic order of $\AB^*$. \\
4. For every $i$ starting from 1 (and increasing by 1 after every loop) \\
4a.\qquad Run $M$ on $w _1, w_2, ..., w_i$ for $i$ steps. \\
4b.\qquad Accept iff $M$ accepts at least two words. \\

So we get: \\
If $\langle M \rangle \in \lang$ then: \\
$\langle M, w \rangle$ is a valid encoding of a TM and $|\lang(M)| > 1$. \\
$\Rightarrow \exists w_{j_1}, w_{j_2} \in \lang(M)$ s.t. $M$ accepts $w_{j_1}, w_{j_2}$ (w.l.o.g.) after $k_1, k_2$ steps, respectively. \\
$\Rightarrow$ after $max(j_1, j_2, k_1, k_2)$ iterations $M$ will accept both words. \\
$\Rightarrow M'$ accepts. \\

If $\langle M \rangle \notin \lang$ then: \\
$\langle M, w \rangle$ is not a valid encoding of a TM or $|\lang(M)| \leq 1$. \\
If $\langle M, w \rangle$ is not a valid encoding of a TM - $M'$ rejects. \\
If $|\lang(M)| \leq 1$: \\
$\Rightarrow M$ accepts at most one word. \\
$\Rightarrow M'$ will be stuck in the loop. \\

We got that $M'$ accepts every $w \in \lang$ and doesn't accept every $w \notin \lang$. Hence $\underline{\lang \in \REcl}$. \\
In conclusion, $\lang \in \REcl \backslash \Rcl$. \\

