Formally describe a 2-tape non-deterministic TM that on input $\#1^n\#$ such
that $n > 1$ chooses non-deterministically $i, j$ such that $0 < i, j \leq n$, writes
$1^i\#1^j$ on the second tape and accepts. You may use a model that allows the
head to stay put (S) as well as moving right (R) and left (L). \\

We will denote the required TM by $M$. \\
The formal description of $M$ will be: \\
$ M = (Q, \AB, \Gamma, \de, q_0, q_a, q_r) $ such that: \\
$ Q      = \{q_0, q_i, q_j, q_a, q_r\} $ \\
$ \AB    = \{1, \#\} $ \\
$ \Gamma = \AB \cup \{\sqcup\} $ \\
and $\de$ is defined by:

\begin{figure}[h]
    \centering
    \begin{tikzpicture}[shorten >=1pt,node distance=2cm,on grid,auto]
        \node[state, initial] (q_0)   {$q_0$};
        \node[state] (q_is) [below=of q_0] {$q_{is}$};
        \node[state] (q_i) [below=of q_is] {$q_i$};
        \node[state] (q_js) [below=of q_i] {$q_{js}$};
        \node[state] (q_j) [below=of q_js] {$q_j$};
        \node[state, accepting] (q_a) [below=of q_j] {$q_a$};
        \path[->]
        (q_0)
        edge node {$(\#, \sqcup) \rightarrow (\#,\sqcup), (R,S)$} (q_is)

        (q_is)
        edge node {$(1, \sqcup) \rightarrow (1,1), (R,R)$} (q_i)
        (q_i)
        edge [loop left] node {$(1, \sqcup) \rightarrow (1,1), (R,R)$} (q_i)
        edge [loop right] node {$(1, \sqcup) \rightarrow (1, \sqcup), (R,S)$} (q_i)
        edge node {$(\#, \sqcup) \rightarrow (\#,\#), (L,R)$} (q_js)

        (q_js)
        edge node {$(1, \sqcup) \rightarrow (1,1), (L,R)$} (q_j)
        (q_j)
        edge [loop left] node {$(1, \sqcup) \rightarrow (1,1), (L,R)$} (q_j)
        edge [loop right] node {$(1, \sqcup) \rightarrow (1, \sqcup), (L,S)$} (q_j)
        edge node {$(\#, \sqcup) \rightarrow (\#,\sqcup), (S,S)$} (q_a);

    \end{tikzpicture}
    \label{fig:multiple5}
\end{figure}


Short explanation: \\
1. First we go over the input from left to right and each time we see '1' we choose
to write '1' on the second tape or write nothing and move on.

2. When we reach the end of the input we have written the '$1^i$' part on the second tape, so we write '$\#$'.

3. Then we go over the input from right to left and repeat the process - each time we
see '1' we choose to write '1' on the second tape or write nothing and move on.

4. When we reach the beginning of the tape, at that point we have written '$1^i\#1^j$' on the second tape,
where $i, j$ were chosen non-deterministically.
