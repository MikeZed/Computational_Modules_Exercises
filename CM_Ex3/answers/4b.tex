Prove that if $f$ is computable then $\lang_f \in \Rcl$.

Proof:
$f$ is computable so there exists TM $M$ that computes it,
meaning that for any $x \in \AB^*$ on $M$'s tape, $M$ will run
and halt at some point with $f(x)$ written on its tape.

We will define the following TM $M'$ that decides $L_f$: \\

$M' \text{ on input }  (x, f(x))$: \\
1. If the input is not in a valid format - reject. \\
2. Run $M$ on $x$ (first element of the pair).  \\
3. Compare the output of $M$ (what is written on its tape)
with $f(x)$ (second element of the pair). \\
4. If equal accept, otherwise reject. \\

So we get: \\
- for $(x, f(x)) \in L_f$ we get an input in a valid format and running
$M$ on $x$ will give us $f(x)$ which is equal to the second element of the input pair.
$\rightarrow M'$ accepts.

- for $(x, f(x)) \in L_f$ we get an input not in a valid format or $M(x) \neq f(x)$: \\
If the input is not in a valid format - $M'$ rejects. \\
If $M(x) \neq f(x)$ - $M'$ reject.

Notice that $M$ halts for any input word, therefore $M'$ halts for any input word.

Indeed we get that $M'$ decides $\lang_f$. So we get $L_f \in \Rcl$. As required. \\
