$co\REcl$ is closed under intersection.

The claim is true. \\

Proof: \\
Let there be $\lang_1, \lang_2 \in co\REcl$, we will show that $\lang = \lang_1 \cap \lang_2 \in co\REcl$. \\
$\lang_1, \lang_2 \in co\REcl$ so there are TM $M_1, M_2$ such that
$\lang(M_1) = \overline{\lang_1}, \lang(M_2) = \overline{\lang_2}$. \\
We shall define TM $M$ such that $\lang(M) = \overline{\lang}$:

$M \text{ on input } x$: \\
1. Run $M_1(x)$ and $M_2(x)$ in parallel (step by step). \\
2. If one of them accepts - accept. \\
3. If both reject - reject.

So we get: \\
If $x \in \overline{\lang} = \overline{\lang_1 \cap \lang_2}$ then
$x \in \overline{\lang_1} \vee x \in \overline{\lang_2}$, so $M_1(x)$ or $M_2(x)$ accept $\rightarrow M$ accepts. \\
If $x \notin \overline{\lang} = \overline{\lang_1 \cap \lang_2}$  then
$x \notin \overline{\lang_1} \wedge x \notin \overline{\lang_2}$, so both $M_1(x)$ and $M_2(x)$ reject or stuck in a loop: \\
- If $M_1(x)$ and $M_2(x)$ reject, then from (3) $M$ rejects. \\
- If $M_1(x)$ or $M_2(x)$ rejects and the other one is stuck in a loop, then  $M$ is stuck in a loop. \\
- If $M_1(x)$ and $M_2(x)$ are stuck in a loop, then  $M$ is stuck in a loop. \\

Indeed we get $\lang(M) = \overline{\lang}$. \\
So we get $\lang = \lang_1 \cap \lang_2 \in co\REcl$, as required. \\

