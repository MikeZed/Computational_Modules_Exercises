Let us define: \\ \\
$
    \begin{aligned}
        Size(O(1)) = \{ & \lang : \text{ There exists a circuit ensemble } C=\{C_n\}_{n \in \nat} \\
                        & \text{such that } \lang(C)=\lang \text{ and } |C_n|\in O(1)\}
    \end{aligned}
$ \\ \\
We will present a non-regular $\lang$ such that $\lang \in Size(O(1))$.

We will use the language from question 3c: \\
$\lang=\{w : \exists n \in \nat \text{ s.t. } |w| = n^3\}$ \\
it was proven in 3c that the above language is not regular.

We can construct the following circuit ensemble $C=\{C_n\}_{n \in \nat}$ such that
$\lang(C)=\lang$ and $|C_n|\in O(1\}$:

For input $x=x_1...x_n$, if $n$ is a cube of an integer ($\sqrt[3]{n} \in \nat$), then
the circuit, $C_n$, will return the constant 1, otherwise (if $\sqrt[3]{n} \notin \nat$)
the circuit, $C_n$, will return the constant 0.

Presenting the above as a formula:
$
    C_n(x_1 ... x_n) =
    \begin{cases}
        1 , & \sqrt[3]{n} \in \nat    \\
        0 , & \sqrt[3]{n} \notin \nat \\
    \end{cases}
$

Presenting as a diagram:

\underline{$C_n, \text{ for } \sqrt[3]{n} \in \nat$:}
\begin{center}
    \begin{tikzpicture}[
            level distance=1.5cm,
            level 1/.style={sibling distance=1cm},
            level 2/.style={sibling distance=1.5cm}]
        \node (y) {$y$}
        child { node (T1) {$x_1$} edge from parent[draw=none]}
        child {node {$\ldots$} edge from parent[draw=none]}
        child { node (Tn) {$x_n$} edge from parent[draw=none]}
        child { node (const0) {$1$} };
    \end{tikzpicture}
\end{center}

\underline{$C_n, \text{ for } \sqrt[3]{n} \notin \nat$:}
\begin{center}
    \begin{tikzpicture}[
            level distance=1.5cm,
            level 1/.style={sibling distance=1cm},
            level 2/.style={sibling distance=1.5cm}]
        \node (y) {$y$}
        child { node (T1) {$x_1$} edge from parent[draw=none]}
        child {node {$\ldots$} edge from parent[draw=none]}
        child { node (Tn) {$x_n$} edge from parent[draw=none]}
        child { node (const0) {$0$} };
    \end{tikzpicture}
\end{center}

Indeed, $w \in \lang$ iff $C_{|w|}(w)=1$.
Therefore $\lang(C)=\lang$ and for each circuit in the ensemble we get $|C_n| = 0 \in O(1\}$ since there are no logic
gates in all the circuits in the ensemble.

Note - if we can't use the constants $\{0, 1\}$ in the circuit ensemble, then we can 'create' those
constants from one of the input bits, it that case we will construct the circuit ensemble according
to the following formula:
$x_1 \vee  (\neg x_1) = 1, x_1 \wedge (\neg x_1) = 0$ and then:  \\
$
    C_n(x_1 ... x_n) = \left.
    \begin{cases}
        x_1 \vee  (\neg x_1) ,  & \sqrt[3]{n} \in \nat    \\
        x_1 \wedge (\neg x_1) , & \sqrt[3]{n} \notin \nat \\
    \end{cases}
    \right. , x_1...x_n = x \in \lang
$

In that case we also get $|C_n| = 2 \in O(1)$

In conclusion, we presented a non-regular language - $\lang=\{w : \exists n \in \nat \text{ s.t. } |w| = n^3\}$,
such that $\lang \in Size(O(1))$. As requested.
