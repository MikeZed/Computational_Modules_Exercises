\documentclass{article}

\usepackage{fancyhdr}
\usepackage{extramarks}
\usepackage{xcolor}
\usepackage{amsmath}
\usepackage{amsthm}
\usepackage{amssymb}
\usepackage{amsfonts}
\usepackage{tikz}
\usepackage[plain]{algorithm}
\usepackage{algpseudocode}
\usepackage[left=1in]{geometry}
\usepackage[shortlabels]{enumitem}
\usepackage{transparent}
\usetikzlibrary{automata,positioning}


%%%%%%%%%%%%%%%%%%%%%%%%%%%%%%%%%%%%%%%%%%%%%%%%%%%%%%%%%%%%%%%%%%%%%%%%%
%%%%%%%%%%%%%%%%%%%%%%%%%%%%% PDF COLOR %%%%%%%%%%%%%%%%%%%%%%%%%%%%%%%%%
%%%%%%%%%%%%%%%%%%%%%%%%%%%%%%%%%%%%%%%%%%%%%%%%%%%%%%%%%%%%%%%%%%%%%%%%%
\usepackage{xcolor} \pagecolor[rgb]{0,0,0} \color[rgb]{0.9,0.9,0.9}
%%%%%%%%%%%%%%%%%%%%%%%%% SUPPRESS UNDERFULL HBOX %%%%%%%%%%%%%%%%%%%%%%%
\hbadness = 20000
%%%%%%%%%%%%%%%%%%%%%%%%%%%%%%%%%%%%%%%%%%%%%%%%%%%%%%%%%%%%%%%%%%%%%%%%%
%
% Basic Document Settings 
%
\input{aux_tex/doc_formatting.tex}
\newcommand{\bd}{\textbf}
\newcommand{\hmwkTitle}{Homework\ \#6}
\newcommand{\hmwkClass}{Computational Modules}

\newcommand{\nat}{\mathbb{N}}
\newcommand{\lang}{\mathcal{L}}
\newcommand{\AB}{\Sigma}
\newcommand{\ch}{\sigma}
\newcommand{\de}{\delta}
\newcommand{\empw}{\varepsilon}
\newcommand{\ceq}[1]{\overset{#1}{\thicksim}}
\newcommand{\nceq}[1]{\overset{#1}{\nsim}}

\newcommand{\Rcl}{\mathcal{R}}
\newcommand{\REcl}{\mathcal{R}\mathcal{E}}
\newcommand{\NPcl}{\mathcal{N}\mathcal{P}}
\newcommand{\NPCcl}{\mathcal{N}\mathcal{P}\mathcal{C}}
\newcommand{\Pcl}{\mathcal{P}}
\newcommand{\bigC}{\mathcal{C}}

\newcommand{\TODO}{\textcolor{red}{ TODO }}
\newcommand{\REV}{\textcolor{brown}{ REV }}


% use for comment block, surround block with "\/*" and at the end "*/" 
\long\def\/*#1*/{}


%%%%%%%%%%%%%%%%%%%%%%%%%%%%%%%%%%%%%%%%%%%%%%%%%%%%%%%%%%%%%%%%%%%%%%%%%
% 
\newcommand{\hmwkAuthorName}{\bd{Michael Zhitomirsky}, ID 321962714}

%%%%%%%%%%%%%%%%%%%%%%%%%%%%%%%%%%%%%%%%%%%%%%%%%%%%%%%%%%%%%%%%%%%%%%%%%

%
% Title 
%


\title{
    \textmd{\bd{\hmwkClass:\ \hmwkTitle}}\\
}
\author{\hmwkAuthorName}

\begin{document}

\maketitle

\begin{enumerate}
      \item Let \(\lang\)  be a regular language. For each of the following languages,
            we shall build an NFA that accepts it and then prove its correctness.
            \(\lang\) is regular, therefore there exists a DFA \(A=(Q, \AB, \delta, q_0, F)\)
            that accepts it, we shall use this DFA in all of the following subsections.

            \begin{enumerate}
                  \item 
$\lang''=\{x_1 x_2 . . . x_k : k \in \nat ,x_i \in \AB \text{ for every }
    1 \leq i \leq k \\ \text{ and } \exists y_1,y_2,...y_{2k} \in \AB
    \text { such that } x_1 y_1 y_2 x_2 y_3 y_4...x_k y_{2k-1} y_{2k} \in \lang \}$
\\ \\
We shall construct the following NFA $N''$ for $\lang''$: \\
$N''=(Q, \AB, \de'', S=\{q_0\}, F)$, such that the
transition function is: \\
$\de''(q,\ch)=\{\de(\de(\de(q,\ch),a),b) : \forall a,b \in \AB\},
    \forall q \in Q, \forall \ch \in \AB$. \\
Meaning that $\de''$ is using $\de$ to make one step with $\ch$ and
then 'guess' two steps with all possible symbols in $\AB$.
\\ \\
Now, we shall prove its correctness: \\
We need to show that: $w \in \lang'' \Longleftrightarrow  w \in \lang(N'')$. \\
Note that: \\
$w \in \lang''$\\

(From definition of $\lang''$) \\
$\Longleftrightarrow w \in \{x_1 x_2 . . . x_k : k \in \nat ,x_i \in \AB \text{ for every }
    1 \leq i \leq k \\ \text{ and } \exists y_1,y_2,...y_{2k} \in \AB
    \text { such that } x_1 y_1 y_2 x_2 y_3 y_4...x_k y_{2k-1} y_{2k} \in \lang \}$  \\

(From definition of $A$: $\lang = \lang(A)$) \\
$\Longleftrightarrow \exists k \in \nat ,x_1...x_k \in \AB^*:  w=x_1 x_2 . . . x_k, \\
    \text{ and } \exists y_1,y_2,...y_{2k} \in \AB
    \text { such that } w_{\lang}=x_1 y_1 y_2 x_2 y_3 y_4...x_k y_{2k-1} y_{2k} \in \lang=\lang(A)$ \\

(From definition of $A$) \\
$\Longleftrightarrow \exists q_1,q_2,...q_{3k} \text{ s.t. } $\\
$
    \de(q_0,x_1)=q_1, \de(q_1,y_1)=q_2, \de(q_2,y_2)=q_3, ..., \\
    \de(q_{3k-3},x_{k})=q_{3k-2}, \de(q_{3k-2},y_{2k-1})=q_{3k-1}, \de(q_{3k-1},y_{2k})=q_{3k} \in F
$ \\ \\
(*) Now, from the above and the definition of $\de''$ we get that: \\
$\exists q_1,q_2,...q_{3k} \text{ s.t. } $\\
$
    \de''(q_0,x_1)=\{\de(\de(\de(q_0,x_1),a),b) : \forall a,b \in \AB\} \ni q_3 \text{ } (q_0 \in S), ..., \\
    \de''(q_{3i-3},x_{i})=\{\de(\de(\de(q_{3i-3},x_{i}),a),b) : \forall a,b \in \AB\} \ni q_{3i}, ..., \\
    \de''(q_{3k-3},x_{k})=\{\de(\de(\de(q_{3k-3},x_{k}),a),b) : \forall a,b \in \AB\} \ni q_{3k} \text{ } (q_{3k} \in F)
$ \\ \\

(From definition of $N''$) \\
$\Longleftrightarrow w=x_1 x_2 . . . x_k \in \lang(N'')$ \\
\\
Note that when looking at the proof in the direction of - ($w \in \lang'' \Longleftarrow  w \in \lang(N'')$),
we get transition (*) by choosing  $y_1,y_2,...y_{2k}$ from the specific $a,b \in \AB$ that give us the
accepting branch in the NFA $N''$ for input $w=x_1 x_2 ... x_k$.
\\

In conclusion, we got $w \in \lang'' \Longleftrightarrow  w \in \lang(N'')$. Thus, concluding the proof. \\


                  \item $\lang''=\{xy : yx \in \lang \}$
\\ \\
The idea would be to construct an NFA per state $q \in Q$, $N_q$,
each NFA will check if we can 'break' the word $yx \in \lang$ in state q of $A$.
So each NFA will contain 2 copies of $A - A_{q,1}, A_{q,2}$ such that for
$yx \in \lang$, $x$ will take us from state q in $A_{q,1}$ to $F_{q,1}$
which will be connected with $\empw$-moves to $q_{0,q,2}$ in $A_{q,2}$,
and y will take us from there to state q in $A_{q,2}$.

The main NFA, $N''$, will be a 'union' of the he smaller NFAs ($N_q, \forall q \in Q$).
It will have an initial state group that is the group of the initial states of each of the
smaller NFAs ($N_q, \forall q \in Q$), and that way we will check if word $w$
is some rotation of some word in $\lang$.

Now, we shall give formal definitions for every one of the NFAs: \\

\underline{$\forall q \in Q, N_q$:} \\

$N_q=(Q \times \{1,2\} \times \{q\}, \AB, \de_q, S_q=\{(q, 1, q)\}, F_q=\{(q, 2, q)\})$, \\
such that the transition function is: \\
\[
    \de_q((q', k, q),\ch) = \left.
    \begin{cases}
        \{(\de(q',\ch)\}, k, q)\} , & \ch \in \AB                          \\
        \{(q_0, 2, q)\},            & q' \in F \wedge k=1 \wedge \ch=\empw \\
    \end{cases}
    \right\} , \text{ } k \in \{1,2\}, \ch \in \AB \cup \{\empw\}
\]

\underline{$N''$:} \\

$N''=((Q \times \{1,2\} \times Q), \AB, \de'',
    S''=\{(q_i, 1, q_i) : q_i \in Q\}, F''=\{(q_f, 2, q_f) : q_f \in Q\})$, \\
such that the transition function is: \\
\[
    \de''((q', k, q),\ch) = \de_{q}((q', k, q),\ch),  k \in \{1,2\}, \ch \in \AB \cup \{\empw\}
\]

Now, we shall prove its correctness: \\
We need to show that: $w \in \lang'' \Longleftrightarrow  w \in \lang(N'')$. \\
Note that: \\
$w \in \lang''$\\

(From definition of $\lang''$) \\
$\Longleftrightarrow w \in \{xy : yx \in \lang \}$ \\

(From definition of $A$: $\lang = \lang(A)$) \\
$\Longleftrightarrow \exists x=x_1...x_n,y=y_1...y_m \in \AB^* : w=xy \wedge yx \in \lang = \lang(A)$ \\
\\ \\

(From definition of $A$) \\
$\Longleftrightarrow \exists q_1,q_2,...q_m, q_{m+1}...q_{n+m} \text{ s.t. } $\\
$
    \de(q_0,y_1)=q_1, \de(q_1,y_2)=q_2, ..., \\
    \de(q_{m-1},y_m)=q_m, \de(q_m,x_1)=q_{m+1},...,\\
    \de(q_{n+m-1},x_{n})=q_{n+m} \in F
$ \\

(From definition of $\de''$) \\
$\Longleftrightarrow \exists q_1,q_2,...q_m, q_{m+1}...q_{n+m} \text{ s.t. } $\\
$
    \de''((q_m, 1, q_m), x_1) \ni (q_{m+1}, 1, q_m) \text{ } ((q_m, 1, q_m) \in S''), ..., \\
    \de''((q_{n+m-1}, 1, q_m), x_n) \ni (q_{n+m}, 1, q_m),\\
    \de''((q_{n+m}, 1, q_m), \empw) \ni (q_0, 2, q_m), (q_{n+m} \in F)\\
    \de''((q_0, 2, q_m), y_1) \ni (q_1, 2, q_m), ...,\\
    \de''((q_{m-1}, 2, q_m), y_m) \ni (q_m, 2, q_m) \text{ } ((q_m, 2, q_m) \in F'')
$ \\

(From definition of $N''$) \\
$\Longleftrightarrow w=x_1...x_n y_1... y_m \in \lang(N'')$ \\

In conclusion, we got $w \in \lang'' \Longleftrightarrow  w \in \lang(N'')$. Thus, concluding the proof. \\

                  \item $\lang''=\{x \in \AB^* : \exists y \in \lang', xy \in \lang\} \text{ for any } \lang'$
\\

We shall construct the following NFA $N''$ for $\lang''$:

$N''=(Q \cup \{q_f\}, \AB, \de'', S''=\{q_0\}, F''=\{q_f\})$,

(we assume w.l.o.g. that $Q \cap \{q_f\} \neq \emptyset$) \\
such that the transition function is: \\
\[
    \de''(q,\ch) = \left.
    \begin{cases}
        \{\de(q, \ch)\} , & \ch \in \AB                                                      \\
        \{q_f\},          & \exists y \in \lang' : \hat{\de}(q, y) \in F  \ \wedge \ch=\empw \\
    \end{cases}
    \right\} , \ch \in \AB \cup \{\empw\}
\]

Now, we shall prove its correctness: \\
We need to show that: $w \in \lang'' \Longleftrightarrow  w \in \lang(N'')$. \\
Note that: \\
$w \in \lang''$\\

(From definition of $\lang''$) \\
$\Longleftrightarrow w \in \{x \in \AB^* : \exists y \in \lang', xy \in \lang\}$ \\

(From definition of $A$: $\lang = \lang(A)$) \\
$\Longleftrightarrow \exists x=x_1...x_n \in \AB^*, y=y_1...y_m \in \lang': w=x,  xy \in \lang = \lang(A)$\\

(From definition of $A$) \\
$\Longleftrightarrow \exists q_1,q_2,...q_m, q_{m+1}...q_{n+m} \text{ s.t. } $\\
$
    \de(q_0,x_1)=q_1, \de(q_1,x_2)=q_2, ..., \\
    \de(q_{n-1},x_n)=q_n, \de(q_n,y_1)=q_{n+1},...,\\
    \de(q_{n+m-1},y_{m})=q_{n+m} \in F
$ \\ \\ \\

(From definition of $\de'' \text{ and }
    \de(q_n,y_1)=q_{n+1},...,\de(q_{n+m-1},y_{m})=q_{n+m} \rightarrow \hat{\de}(q_n, y)=q_{n+m}$) \\
$\Longleftrightarrow \exists q_1,q_2,...q_m, q_{m+1}...q_{n+m} \text{ s.t. } $\\
$
    \de''(q_0,x_1) \ni q_1  \text{ } ((q_0) \in S''), ..., \\
    \de''(q_{n-1},x_n) \ni q_n, \\
    \hat{\de}(q_n, y)=q_{n+m} \in F \rightarrow \de''(q_n,\empw)=\{q_f\}, q_f \in F''
$ \\

(From definition of $N''$) \\
$\Longleftrightarrow w=x_1...x_n \in \lang(N'')$ \\

In conclusion, we got $w \in \lang'' \Longleftrightarrow  w \in \lang(N'')$. Thus, concluding the proof. \\


            \end{enumerate}

            \pagebreak

      \item We shall present a regular expression for the following languages and give a short
            explanation for their correctness:

            \begin{enumerate}
                  \item If $\lang \in \REcl$ and $\lang \leq_m \overline{\lang}$, then $\lang \in \Rcl$ \\

The claim is true. \\

Proof: \\
From $\lang \leq_m \overline{\lang}$ we also get that $\overline{\lang} \leq_m \lang$.  \\
Then, from $\lang \in \REcl$, we get that $\overline{\lang} \in \REcl$ meaning $\lang \in co\REcl$. \\
Hence $\lang \in \REcl \wedge \lang \in co\REcl \Longrightarrow \lang \in \REcl \cap co\REcl = \Rcl$. \\

                  \item If $\lang \in co\REcl$ and $\lang \leq_m \overline{\lang}$, then $\lang \in \Rcl$ \\
The claim is true. \\

Proof: \\
From $\lang \in co\REcl$ we get that $\overline{\lang} \in \REcl$.  \\
Then, from $\lang \leq_m \overline{\lang}$, we get that $\lang \in \REcl$. \\
Hence $\lang \in \REcl \wedge \lang \in co\REcl \Longrightarrow \lang \in \REcl \cap co\REcl = \Rcl$. \\

                  \item The complement of $\lang((0 \cup 10 \cup 110)^*(\empw \cup 1 \cup 11))$

Note that the first part ($(0 \cup 10 \cup 110)^*$) gives us an arbitrary number of 0's
and sequences that contain 1 or 11 and followed by an arbitrary number of 0's (\bd{at least one 0}).
And the second part ($\empw \cup 1 \cup 11$) gives us the empty word or 1 or 11.

Therefore the words in the language have sequences of at most two 1's
and the complement of language will be words that are arbitrary strings that contain at least three 1's:

$R_\lang = (0 \cup 1)^*111(0 \cup 1)^*$ \\
            \end{enumerate}

            \pagebreak

      \item We shall prove that the following languages are not regular:
            \begin{enumerate}
                  \item If $\lang_1, \lang_2 \in \Pcl$ are non-trivial ($\lang_1, \lang_2 \notin \{\emptyset, \AB^*\}$), then there exists a polynomial shrinking
reduction from $\lang_1$ to $\lang_2$. \\
The claim is true. \\

Proof: \\
$\lang_2$ is not trivial, hence $\exists w_1, w_2: w_1 \in \lang_2 \wedge w_2 \notin \lang_2$.
Therefore we can define the following reduction from $\lang_1$ to $\lang_2$ - $f: \AB^* \rightarrow \AB^*$:
\[
    f(x) =
    \begin{cases}
        w_1 , & x \in \lang_1    \\
        w_2 , & x \notin \lang_1 \\
    \end{cases}
\]

Notice that $f$ is computable in polynomial time since $\lang_1 \in \Pcl$ (meaning there is a TM
$M$ that decides $\lang_1$ in a polynomial time, therefore we can use it to check whether
$x \in \lang_1$ or $x \notin \lang_1$ in a polynomial time and return the relevent word) and $\forall x \in \AB^*$:

If $x \in \lang_1$
$\Longrightarrow f(x) = w_1 \in \lang_2$

If $x \notin \lang_1$
$\Longrightarrow f(x) = w_2 \notin \lang_2$

So we get that $\lang_1 \leq_p \lang_2$. \\

Now we will show that $f$ is a polynomial shrinking reduction: \\
Notice that for $n_0 = \max\{|w_0|, |w_1|\} + 1 \in \nat$ we get that for all $x \in \AB^* \wedge |x| \geq n_0$:

If $x \in \lang_1 \Longrightarrow f(x) = w_1$
$\Longrightarrow |f(x)| = |w_1| < n_0 \leq |x|$

If $x \notin \lang_1 \Longrightarrow f(x) = w_2$
$\Longrightarrow |f(x)| = |w_2| < n_0 \leq |x|$

So there exists $n_0 \in \nat$ such that for every $x \in \AB^*$ it holds that if $n_0 \leq |x|$ then $|f(x)| < |x|$,
meaning that $f$ is a polynomial shrinking reduction. As required.
                  \item Prove that if $\AB^* \in \bigC$ then $\lang_\bigC \notin co\REcl$

Proof: \\
Notice that $\AB^* \in \bigC \subset \REcl$, then $\AB^* \notin \overline{\bigC} = \REcl \backslash \bigC$. \\
Now, because $\overline{\bigC} \subseteq \REcl$, $\overline{\bigC} \neq \emptyset$ ($\bigC \neq \REcl$) and $\overline{\bigC} \neq \REcl$ ($\bigC \neq \emptyset$),
we get from the previous section that $\overline{ACC} \leq_m \lang_{\overline{\bigC}}$. \\
Hence $ACC \leq_m \overline{\lang}_{\overline{\bigC}}$ and then from $ACC \in \REcl \backslash \Rcl$ we get $ACC \notin co\REcl \Rightarrow \overline{\lang}_{ \overline{\bigC}} \notin co\REcl$. \\
It holds that: \\
$\overline{\lang}_{\overline{\bigC}} = \overline{\{\langle M \rangle : \lang(M) \in \overline{\bigC}\}} \\
    = \{\langle M \rangle : \langle M \rangle \text{ is not valid encoding of a TM }\} \cup \{\langle M \rangle : \lang(M) \in \bigC\}  \\
    = \{\langle M \rangle : \langle M \rangle \text{ is not valid encoding of a TM }\} \cup \lang_\bigC$ \\

Notice that $\lang_{M\_invalid} = \{\langle M \rangle : \langle M \rangle \text{ is not valid encoding of a TM }\} \in \Rcl$ \\
since we can build a TM that gets $w \in \AB^*$ and checks if it is a valid encoding $w = \langle M \rangle$, if it is ($w \notin \lang_{M\_invalid}$) - it rejects,
otherwise ($w \in \lang_{M\_invalid}$) it accepts.

So we got that: \\
1. $\overline{\lang}_{\overline{\bigC}} = \lang_{M\_invalid} \cup \lang_\bigC$. \\
2. $\lang_{M\_invalid} \in \Rcl \Rightarrow \lang_{M\_invalid} \in co\REcl$. \\
3. $\overline{\lang}_{\overline{\bigC}} \notin co\REcl$. \\

Since $co\REcl$ is closed to union (just like $\REcl$ is closed to union, proof is after this section) we get that $\lang_\bigC \notin co\REcl$, \\
because otherwise we will get from (1) and (2) that $\overline{\lang}_{\overline{\bigC}} \in co\REcl$ in contradiction with (3). \\

In conclusion, we get that $\lang_\bigC \notin co\REcl$. As required

\begin{center}
    \noindent\rule{4cm}{0.4pt}
\end{center}
\underline{Proof that $co\REcl$ is closed under union}: \\

Let there be $\lang_1, \lang_2 \in co\REcl$, we will show that $\lang = \lang_1 \cup \lang_2 \in co\REcl$. \\
$\lang_1, \lang_2 \in co\REcl$ so there are TM $M_1, M_2$ such that
$\lang(M_1) = \overline{\lang_1}, \lang(M_2) = \overline{\lang_2}$. \\
We shall define TM $M$ such that $\lang(M) = \overline{\lang}$:

$M$ on input $x$: \\
1. Run $M_1(x)$ and $M_2(x)$ in parallel (step by step). \\
2. If both of them accepts - accept. \\
3. Reject.

So we get: \\
If $x \in \overline{\lang} = \overline{\lang_1 \cup \lang_2} = \overline{\lang_1} \cap \overline{\lang_2}$ then
$x \in \overline{\lang_1} \wedge x \in \overline{\lang_2}$, so $M_1(x)$ and $M_2(x)$ accept $\Rightarrow M$ accepts. \\
If $x \notin \overline{\lang} = \overline{\lang_1 \cup \lang_2} = \overline{\lang_1} \cap \overline{\lang_2}$  then
$x \notin \overline{\lang_1} \vee x \notin \overline{\lang_2}$, so $M_1(x)$ or $M_2(x)$ reject or stuck in a loop: \\
- If $M_1(x)$ or $M_2(x)$ reject, then $M$ rejects. \\
- If $M_1(x)$ and $M_2(x)$ is stuck in a loop, then $M$ is stuck in a loop. \\

Indeed we get $\lang(M) = \overline{\lang}$. \\
So we get $\lang = \lang_1 \cup \lang_2 \in co\REcl$, as required. \\


                  \item $\Rcl$ is closed under Kleene star. \\
            \end{enumerate}

            \pagebreak

      \item A TM is {\it expected poly-time } if there exists $p \in poly$ such that for every x the
expected running time of $M(x)$ is at most $p(|x|)$. \\
Let us define: \\
$ZPP = \{\lang \subseteq \AB^*: \text{ there exists an expected poly-time TM that decides } \lang\}$ \\
Prove that $ZPP = RP \cap coRP$.

Proof: \\
We need to show that: \\
1. $ZPP \subseteq RP \cap coRP$ \\
2. $ZPP \supseteq RP \cap coRP$ \\

\underline{$ZPP \subseteq RP \cap coRP$:} \\
Let $\lang \in ZPP$, so there exists an expected poly-time TM $M$ with poly-time $p$, that decides $\lang$. \\
We will show that $\lang \in RP \cap coRP$, by showing that $\lang \in RP \wedge \lang \in coRP$:

\underline{$\lang \in RP$:} \\
We shall define probabilistic TM $M_{RP}$ for $\lang$.  \\
$M_{RP}$ on input $x$:
\begin{enumerate}[1., itemsep=5pt]
    \item Run $M$ on $x$ for $2p(|x|)$ steps.
    \item If it stops, answer as $M(x)$ answers.
    \item Reject.
\end{enumerate}
We denote by $T_{M,x}$ (random variable) the runtime of $M$ on input $x$ until it stops.

So we get $\forall x \in \AB^*$: \\
If $x \in \lang$ \\
$\Longrightarrow M$ accepts $x$ in expected poly-time $p(|x|)$ \\
$\Longrightarrow M_{RP}$ rejects $x$ only if $T_{M,x} > 2p(|x|)$ \\
$\Longrightarrow M_{RP}$ rejects $x$ with probability (Markov's Inequelity):
\[ P(T_{M,x} \geq 2p(|x|)) \leq E[T_{M,x}] / 2p(|x|) = p(|x|) / 2p(|x|) = 1/2 \]
$\Longrightarrow M_{RP}$ accepts $x$ with probability $\geq 1/2$

If $x \notin \lang$ \\
$\Longrightarrow M$ rejects $x$ \\
$\Longrightarrow M_{RP}$ rejects $x$ whether $M(x)$ will stop or not \\

Notice that the TM's runtime is polynomial $= O(2p(|x|)) = O(p(|x|))$ \\
Therefore $\lang \in RP$. \\

\underline{$\lang \in coRP$:} \\
We shall define probabilistic TM $M_{coRP}$ for $\lang$
$M_{coRP}$ on input $x$:
\begin{enumerate}[1., itemsep=5pt]
    \item Run $M$ on $x$ for $2p(|x|)$ steps.
    \item If it stops, answer as $M(x)$ answers.
    \item Accept.
\end{enumerate}
We denote by $T_{M,x}$ (random variable) the runtime of $M$ on input $x$ until it stops.

\pagebreak
So we get $\forall x \in \AB^*$: \\
If $x \in \lang$ \\
$\Longrightarrow M$ accepts $x$ \\
$\Longrightarrow M_{coRP}$ accepts $x$ whether $M(x)$ will stop or not

If $x \notin \lang$ \\
$\Longrightarrow M$ rejects $x$ in expected poly-time $p(|x|)$ \\
$\Longrightarrow M_{coRP}$ accepts $x$ only if $T_{M,x} > 2p(|x|)$ \\
$\Longrightarrow M_{coRP}$ accepts $x$ with probability (Markov's Inequelity):
\[ P(T_{M,x} \geq 2p(|x|)) \leq E[T_{M,x}] / 2p(|x|) = p(|x|) / 2p(|x|) = 1/2 \]
$\Longrightarrow M_{coRP}$ accepts $x$ with probability $\leq 1/2$ \\

Notice that the TM's runtime is polynomial $= O(2p(|x|)) = O(p(|x|))$ \\
Therefore $\lang \in coRP$. \\

Hence $\lang \in RP \cap coRP \Rightarrow ZPP \subseteq RP \cap coRP$. \\

\underline{$ZPP \supseteq RP \cap coRP$:} \\
Let $\lang \in RP \cap coRP$, so $\lang \in RP \wedge \lang \in coRP$. Therefore there exists
probabilistic TM $M_{RP}, M_{coRP}$ such that: \\
- If $x \in    \lang: M_{RP}$ accepts $x$ with probability $\geq 1/2$, $M_{coRP}$ always accepts $x$,  \\
- If $x \notin \lang: M_{RP}$ always rejects $x$, $M_{coRP}$ accepts $x$ with probability $\leq 1/2$ \\
(*) Therefore if $M_{RP}$ accepts $x$, then definitely $x \in \lang$
and if $M_{coRP}$ rejects $x$, then definitely $x \notin \lang$. \\

We define TM $M$ for language $\lang$: \\
$M$ on input $x$:
\begin{enumerate}[1., itemsep=5pt]
    \item Run $M_{RP}$ on $x$ with new random string $r$, if it accepts - accept.
    \item Run $M_{coRP}$ on $x$ with new random string $r$, if it rejects - reject.
    \item Return to step 1.
\end{enumerate}

Notice that $M$'s answer is always correct, since we run both TMs until we get an answer according to (*).
Since the probability of error for $M_{RP}, M_{coRP}$ in each case is $1/2$,
each of them is expected to run twice, since their runtime is polynomial we get that
the algorithm's runtime is expected poly-time.

Hence $\lang \in ZPP \Rightarrow ZPP \supseteq RP \cap coRP$. \\
so we get that $ZPP = RP \cap coRP$. As required.


      \item Let us define: \\ \\
$
    \begin{aligned}
        Size(O(1)) = \{ & \lang : \text{ There exists a circuit ensemble } C=\{C_n\}_{n \in \nat} \\
                        & \text{such that } \lang(C)=\lang \text{ and } |C_n|\in O(1)\}
    \end{aligned}
$ \\ \\
We will present a non-regular $\lang$ such that $\lang \in Size(O(1))$.

We will use the language from question 3c: \\
$\lang=\{w : \exists n \in \nat \text{ s.t. } |w| = n^3\}$ \\
it was proven in 3c that the above language is not regular.

We can construct the following circuit ensemble $C=\{C_n\}_{n \in \nat}$ such that
$\lang(C)=\lang$ and $|C_n|\in O(1\}$:

For input $x=x_1...x_n$, if $n$ is a cube of an integer ($\sqrt[3]{n} \in \nat$), then
the circuit, $C_n$, will return the constant 1, otherwise (if $\sqrt[3]{n} \notin \nat$)
the circuit, $C_n$, will return the constant 0.

Presenting the above as a formula:
$
    C_n(x_1 ... x_n) =
    \begin{cases}
        1 , & \sqrt[3]{n} \in \nat    \\
        0 , & \sqrt[3]{n} \notin \nat \\
    \end{cases}
$

Presenting as a diagram:

\underline{$C_n, \text{ for } \sqrt[3]{n} \in \nat$:}
\begin{center}
    \begin{tikzpicture}[
            level distance=1.5cm,
            level 1/.style={sibling distance=1cm},
            level 2/.style={sibling distance=1.5cm}]
        \node (y) {$y$}
        child { node (T1) {$x_1$} edge from parent[draw=none]}
        child {node {$\ldots$} edge from parent[draw=none]}
        child { node (Tn) {$x_n$} edge from parent[draw=none]}
        child { node (const0) {$1$} };
    \end{tikzpicture}
\end{center}

\underline{$C_n, \text{ for } \sqrt[3]{n} \notin \nat$:}
\begin{center}
    \begin{tikzpicture}[
            level distance=1.5cm,
            level 1/.style={sibling distance=1cm},
            level 2/.style={sibling distance=1.5cm}]
        \node (y) {$y$}
        child { node (T1) {$x_1$} edge from parent[draw=none]}
        child {node {$\ldots$} edge from parent[draw=none]}
        child { node (Tn) {$x_n$} edge from parent[draw=none]}
        child { node (const0) {$0$} };
    \end{tikzpicture}
\end{center}

Indeed, $w \in \lang$ iff $C_{|w|}(w)=1$.
Therefore $\lang(C)=\lang$ and for each circuit in the ensemble we get $|C_n| = 0 \in O(1\}$ since there are no logic
gates in all the circuits in the ensemble.

Note - if we can't use the constants $\{0, 1\}$ in the circuit ensemble, then we can 'create' those
constants from one of the input bits, it that case we will construct the circuit ensemble according
to the following formula:
$x_1 \vee  (\neg x_1) = 1, x_1 \wedge (\neg x_1) = 0$ and then:  \\
$
    C_n(x_1 ... x_n) = \left.
    \begin{cases}
        x_1 \vee  (\neg x_1) ,  & \sqrt[3]{n} \in \nat    \\
        x_1 \wedge (\neg x_1) , & \sqrt[3]{n} \notin \nat \\
    \end{cases}
    \right. , x_1...x_n = x \in \lang
$

In that case we also get $|C_n| = 2 \in O(1)$

In conclusion, we presented a non-regular language - $\lang=\{w : \exists n \in \nat \text{ s.t. } |w| = n^3\}$,
such that $\lang \in Size(O(1))$. As requested.



\end{enumerate}

\bd{Bonus.} \\
We will define a state per equivalence class of $\lang$
transition function according to the equivalence class
accepting state if

\end{document}