Input: sets $A_1, ..., A_n$, and a number $k$. \\
Question: do there exist $k$ mutually disjoint sets $A_{i_1} , ..., A_{i_l}$

We need to show that: \\
$\lang = \{\langle A = \{A_1, ..., A_n\}, k \rangle: \text{ there is a subset
        of A of k mutually disjoint sets}\} \in \NPCcl$ \\

Proof: \\
We will show that: \\
1. $\lang \in \NPcl$ \\
2. $\forall \lang' \in \NPcl: \lang' \leq_p \lang$ by showing that $IS \leq_p \lang$ (we can do this since $IS \in \NPCcl$) \\

First we show that $\lang \in \NPcl$, by building a polynomial verifier $V$ for it. \\
$V$ for input $(\langle A = \{A_1, ..., A_n\}, k \rangle, w)$:
\begin{enumerate}[1., itemsep=5pt]
    \item Verify that $\langle A = \{A_1, ..., A_n\}, k \rangle$ is a valid encoding of a set of sets and a number, if it's not - reject.
    \item Verify that $w$ is a valid encoding of a subset of size $k$ in $A$ - $A_{i_1} , ..., A_{i_l}$, if it's not - reject.

    \item For every 2 sets in the subset - $A_j, A_l$
    \item \qquad If $A_j \cap A_l \neq \emptyset$ - reject. (Check this by comparing every element in $A_j$ with every element in $A_l$)

    \item Accept.

\end{enumerate}

Notice that steps (3-4) check whether the subset is indeed a mutually disjoint sets of size $k$.\\

So we get: \\
If $\langle A = \{A_1, ..., A_n\}, k \rangle \in \lang$ \\
$\Rightarrow \langle A = \{A_1, ..., A_n\}, k \rangle$ is a valid encoding of a set of sets and a number
and there is a subset of $A$ of size $k$ of mutually disjoint sets - $A_{i_1} , ..., A_{i_l}$.\\
$\Rightarrow $ For $w=\langle A_{i_1} , ..., A_{i_l} \rangle$, $V$ will verify that indeed every 2 sets in the
subset are mutually disjoint and accept. \\
$\Rightarrow $ There exists $w \in \AB^*$ such that $V$ accepts $(\langle A = \{A_1, ..., A_n\}, k \rangle, w)$. \\

If $\langle A = \{A_1, ..., A_n\}, k \rangle \notin \lang$ \\
$\Rightarrow \langle A = \{A_1, ..., A_n\}, k \rangle$ is not a valid encoding of a set of sets and a number
or there doesn't exist a subset of $A$ of size $k$ of mutually disjoint sets. \\
If $\langle A = \{A_1, ..., A_n\}, k \rangle$ is not a valid encoding of a set of sets and a number $\Rightarrow V$ rejects. \\
If there doesn't exist a subset of $A$ of $k$ mutually disjoint sets \\
$\Rightarrow V$ rejects $(\langle A = \{A_1, ..., A_n\}, k \rangle, w)$ for every $w \in \AB^*$. \\

Also $V$'s runtime on input $(\langle A = \{A_1, ..., A_n\}, k \rangle, w)$
(where $|\langle A = \{A_1, ..., A_n\}, k \rangle|=n$) is: \\
- For step (1): $O(n))$  \\
- For step (2): $O(n))$  \\
- For steps (3-4): (number of sets comparisons) $\cdot$ (number of elements to compare) = \\
= $O((n-1)+(n-2)+...+1) \cdot O(|A_j| \cdot |A_l|)$ (for $A_j, A_l \in A) = O(n^2) \cdot O(n^2) = O(n^4)$ \\
- For step (5): $O(1)$ \\
Thus, the total runtime of $V$ on input  $(\langle A = \{A_1, ..., A_n\}, k \rangle, w)$ is polynomial \\
in $|\langle A = \{A_1, ..., A_n\}, k \rangle|=n$, so \underline{$\lang \in \NPcl$}. \\

Now, we will show that there is a polynomial reduction from $IS$ to $\lang$: \\
We define the following reduction $f: \AB^* \rightarrow \AB^*$:

$
    f(\langle G=(V,E), k \rangle) =
    \begin{cases}
        F = \langle A=\{1\}, 5 \rangle,                                      & \langle G=(V,E), k \rangle \text{ is not a valid encoding } \\
        \langle A=\{A_v = \{\{u,v\} \in E: u \in V\} | v \in V\}, k \rangle, & else                                                        \\
    \end{cases}
$
Notice that  $F = \langle A=\{1\}, 5 \rangle \notin \lang$. \\

Notice that $f$ is computable in polynomial time in $|\langle G=(V,E), k \rangle|$, since it mainly goes over the edges in the graph $G$.

Also $\forall x = \langle G=(V,E), k \rangle \in \AB^*$:

If $x = \langle G=(V,E), k \rangle \in IS$ \\
$\Rightarrow G$ is an undirected graph with IS of size $k$ \\
$\Rightarrow f(\langle G=(V,E), k \rangle) = \langle A=\{A_v = \{\{u,v\} \in E: u \in V\} | v \in V\}, k \rangle$, and A has a subset
of a size $k$ that is mutually disjoint (the sets in this subset are the ones that came from the vertices in the IS,
each of them is equal to $\emptyset$ since there are no edges that are connected to these vertices, hence the intersection between them is $\emptyset$) \\
$\Rightarrow f(\langle G=(V,E), k \rangle) \in \lang$ \\

If $x = \langle G=(V,E), k \rangle \notin IS$ \\
$\Rightarrow \langle G=(V,E), k \rangle$ is not valid encoding or $G$ is an undirected graph with no IS of size $k$. \\
If $\langle G=(V,E), k \rangle$ is not valid encoding \\
$\Rightarrow f(\langle G=(V,E), k \rangle) = F = \langle A=\{1\}, 5 \rangle \notin \lang$ \\
If $G$ is an undirected graph with no IS of size $k$ \\
$\Rightarrow $ If we assume that $f(\langle G=(V,E), k \rangle) = \langle A=\{A_v = \{\{u,v\} \in E: u \in V\} | v \in V\}, k \rangle \in \lang$ \\
then we have a subset of A of size k of mutually disjoint sets, meaning that for every 2 sets in the subset - $A_j, A_l$,
we have that $A_j \cap A_l = \emptyset \Rightarrow \{\{u,j\} \in E: u \in V\} \cap \{\{u,l\} \in E: u \in V\} = \emptyset$
which means that there are no edges that connect vertices $j$ and $l$. So we get that the vertices that the subset comes from, are an
IS of size $k$, which is a contradiction. \\
$\Rightarrow f(\langle G=(V,E), k \rangle) \notin \lang$ \\

So we get \underline{$IS \leq_p \lang$}. \\
In conclusion, $\lang \in \NPCcl$. As required.
